% Dokumentklasse und Spracheinstellung
\documentclass[12pt, oneside, paper=A4, DIV=15, BCOR=0mm, abstract=true, headings=small]{scrartcl}
\usepackage[ngerman, english]{babel}
\usepackage[utf8]{inputenc}

% Schriftart
%LModern
\usepackage{lmodern}
%Libertine
%\usepackage{libertinus}
%\usepackage{libertinust1math}

\usepackage[T1]{fontenc}

% Kopf- und Fusszeilen mit scrlayer-scrpage steuern
% Trennungslinie oben und unten
\usepackage[headsepline, footsepline]{scrlayer-scrpage}


% fontsize for sections
\usepackage{titlesec}

\titleformat*{\section}{\large\bfseries}
%\titleformat*{\subsection}{\large\bfseries}
%\titleformat*{\subsubsection}{\large\bfseries}


% Mathe, Symbole, EInheitendarstellung, Chemie
\usepackage{amsmath}
% Grouping Figure/Equation Numbering by Section
%\numberwithin{equation}{section}
%\numberwithin{figure}{section}
\usepackage{amssymb}
\usepackage{esdiff}
\usepackage{amsxtra}
\usepackage{eurosym}
\usepackage{siunitx}  
\sisetup{locale=DE}
\usepackage[version=4]{mhchem}
\usepackage[dvipsnames]{xcolor}

% Typographie
\usepackage[auto]{microtype}
\clubpenalty = 10000
\widowpenalty = 10000
\displaywidowpenalty = 10000

% Einbindung von Bildern, Tabellen, pdf-Seiten, Quellcode
\usepackage{graphicx}
\usepackage{multirow, multicol, booktabs}
\usepackage{threeparttable}
\usepackage{longtable}
\usepackage{rotating}
\usepackage{ltablex}
\usepackage{subfig}
\captionsetup[subtable]{position=top}
\usepackage{pdfpages}
\usepackage{listings}

% Darstellung von URL
\usepackage{url}
\urlstyle{same}

% Fussnoten, auch für Tabellen
\usepackage{footnote}
\makesavenoteenv{tabular} 

% Pakete für Kontrolle und Review
\usepackage{todonotes}
\usepackage{blindtext}

% Darstellung der Literaturangaben
\usepackage[
backend=biber,
style=iso-numeric,
citestyle=numeric-comp,
maxbibnames=2,
firstinits=true
]{biblatex}

\renewcommand*{\labelnamepunct}{\addcolon\addspace}

% Speicherort der Literaturangaben (*.bib Datei)
\bibliography{literatur/literaturdatenbank}

% Fussnoten
% Markierung in der Fußnote selbst weder hochgestellt noch kleiner gesetzt
% \deffootnote{1em}{1em}{\thefootnotemark\ }
% linksbündige Fußnotenmarkierungen
\deffootnote{1.5em}{1em}{%
  \makebox[1.5em][l]{\thefootnotemark}%
}

% Fussnoten nicht umbrechen
\interfootnotelinepenalty=10000

% Gestaltung der Bildunterschrift und Tabellenüberschirften sowie Titelseitenangaben
\addtokomafont{caption}{\small}
\setkomafont{captionlabel}{\sffamily\bfseries}
\setkomafont{subject}{\Large\bfseries}
\setkomafont{author}{\normalfont}
\setkomafont{date}{\normalfont}
\setkomafont{publishers}{\normalfont}


\makeatletter
% \renewcommand{\l@section}{\@dottedtocline{1}{0em}{0em}}
% \renewcommand{\l@subsection}{\@dottedtocline{2}{4ex}{3.6em}}
% \g@addto@macro{\@afterheading}{\vspace{-0.25\baselineskip}}
\renewcommand{\l@section}{\@dottedtocline{1}{0ex}{1.8em}}
\renewcommand{\l@subsection}{\@dottedtocline{2}{1.8em}{2.7em}}
% \g@addto@macro{\@afterheading}{\vspace{-0.25\baselineskip}}
\makeatother


% \renewcaptionname{ngerman}{\figurename}{Abb.}
% \renewcaptionname{ngerman}{\tablename}{Tab.} 

% Tabellenumgebungen mit Schriftgröße 10 und 7
\newenvironment{tabular10}{%
  \fontsize{10}{12}\selectfont\tabular
}{%
  \endtabular
}

\newenvironment{tabular7}{%
  \fontsize{7}{12}\selectfont\tabular
}{%
  \endtabular
}

% Verweise und Refernezen, pdf-Eisntellungen
% Angaben ggf. aktualisieren!
\usepackage[
pdftitle={StudyNotes},
pdfsubject={},
pdfauthor={},
pdfkeywords={},  
% Links nicht einrahmen
hidelinks
]{hyperref}
\usepackage[german, english]{cleveref}

\graphicspath{ {./bilder/} }
